% select document class
\documentclass[11pt,a4paper]{article}

% include packages
\usepackage[T1]{fontenc}
\usepackage[ngerman]{babel}
%\usepackage[autostyle=true]{csquotes}
\usepackage{amsmath}
\usepackage{amssymb}
\usepackage{mathtools}
\usepackage{graphicx}
\usepackage{hyperref}
\usepackage{siunitx}
\usepackage{geometry}

% setup for packages
\hypersetup{colorlinks=true, linkcolor=black, citecolor=black}
\geometry{
 a4paper,
 total={170mm,257mm},
 left=20mm,
 top=20mm,
 }
\graphicspath{{figs/}}
\newcommand{\citelink}[2]{\hyperlink{cite.\therefsection @#1}{#2}}

% generate title
\title{\textbf{Computerphysik SS 2023 \\ Hausaufgabe 1}}
\author{Gabriel Remiszewski und Christian Fischer}
\date{1.Mai 2023}

% document
\begin{document}

\maketitle

\section*{Madelung-Energie des Natriumchlorid-Kristalls}\label{sec:energie}

1. Im Folgenden soll die elektromagnetische Energie eines einzelnen Gitterbausteins in einem dreidimensionalen Natriumchlorid-Kristall
(NaCl) mit einer relativen Genauigkeit von $10^{-5}$ berechnet werden. Für einen einzelnen Gitterbaustein $i$ geschieht dies durch
Aufsummation der Einzelbeiträge
\begin{equation*}
    V_{ij} = \frac{1}{4\pi\epsilon_0}\frac{e_i e_j}{r_{ij}}, \quad i = (i_1,i_2,i_3), \quad j = (j_1,j_2,j_3).
\end{equation*}
Hierbei bezeichnen $e_{i,j} = \pm e$ die Ladungen an den entsprechenden Gitterplätzen $i,j$ mit der Elementarladung $e$
und es ist
\begin{equation*}
    r_{ij} = a \sqrt{(i_1 - j_1)   ^2 + (i_2 - j_2)    ^2 + (i_3 - j_3)    ^2}
\end{equation*}
mit dem Gitterabstand $a$. Ein Natriumchlorid-Kristall setzt sich aus alternierenden Natrium-Ionen (positiv geladen) und
Chlorid-Ionen (negativ geladen) zusammen \cite{key1}. Demnach ist ein unendlich ausgedehnter, dreidimenionaler Natriumchlorid-Kristall
insgesamt neutral geladen. Aufgrund dieser Kristallstruktur ist es naheliegend, den Kristall in Elementarzellen aufzuteilen \cite{key2}. Eine solche Elementarzelle
für einen Natriumchlorid-Kristall ist in Abbildung \ref{fig:nacl} dargestellt.
\begin{figure}[htbp]
    \centering
    \includegraphics[width=0.4\textwidth,scale=0.4]{Cube}
    \caption[Elementarzelle eines Natriumchlorid-Kristalls.]{Elementarzelle eines Natriumchlorid-Kristalls.}\label{fig:nacl}
\end{figure}
Hierbei stellen die blauen Punkte negative und die roten Punkte positive Ladungen dar. Den Ladungen werden die Gewichte $1, \frac{1}{2}, \frac{1}{4}$ oder $\frac{1}{8}$
zugeteilt, je nachdem ob sie sich im Inneren, auf einer Seitenfläche, auf einer Kante oder auf einer Ecke des Würfels befinden. Dann verschwindet die Gesamtladung einer solchen
Elementarzelle. Nach H. M. Evjen \cite{key2} wird dann über die Potentiale der individuellen Zellen und nicht über die Potentiale der individuellen Ionen summiert, um die elektromagnetische Energie
eines einzelnen Gitterbausteins zu berechnen. Nach der sogenannten Evjen-Methode erfolgt dies über eine Summation über die Potentiale der einzelnen Ionen, wobei deren Ladung (wie oben bereits aufgeführt) gewichtet wird, je nachdem ob sie sich im Inneren,
auf einer Seitenfläche, auf einer Kante oder auf einer Ecke einer Elementarzelle befinden. Dieses Vorgehen ist äquivalent zu einer Summation über die Potentiale der Elementarzellen. Diese Methode
wird auch später im Code implementiert, wobei die Gewichtung der Ladungen durch eine entsprechende Benutzung von if-Statements erzielt wird.\newline Nun ist es unabdingbar, eine geeignete Summations-Formel zu finden.
Eine erste Vereinfachung des Problems besteht darin, den Gitterbaustein, auf den sich die zu berechnende elektromagnetische Energie bezieht, auf den Gitterplatz $(0,0,0)$ zu setzen. Für diesen
Gitterbaustein wird o.B.d.A ein Natrium-Ion (positiv geladen) (siehe Abbildung \ref{fig:nacl}) gewählt. Dieser Gitterbaustein stellt dann das Zentrum einer Elementarzelle dar. Damit vereinfachen sich die Einzelbeiträge der elektromagnetischen Energie zu
\begin{equation*}
    V_{i,j,k} = \frac{e}{4\pi\epsilon_0}\frac{e_{i,j,k}}{r_{i,j,k}}, \quad r_{i,j,k} = a \sqrt{i^2 + j^2 + k^2}
\end{equation*} wobei die Indizes $i,j,k$ den Gitterplatz eines anderen Ions beschreiben. Dann folgt sofort
\begin{equation*}
    V = \sum_{i,j,k=-n}^{n} V_{i,j,k} = \sum_{i,j,k=-n}^{n} \frac{e}{4\pi\epsilon_0} \frac{e_{i,j,k}}{a \sqrt{i^2 + j^2 + k^2}} = \frac{1}{4\pi\epsilon_0} \frac{e^2}{a} \sum_{i,j,k=-n}^{n} \frac{(-1)   ^{i+j+k}}{\sqrt{i^2 + j^2 + k^2}} = -\frac{1}{4\pi\epsilon_0} \frac{e^2}{a} \alpha,
\end{equation*} wobei im vorletzten Schritt die alternierende Abfolge von positiven und negativen Ladungen im\newline Natriumchlorid-Kristall ausgenutzt wurde (negativer Beitrag,
wenn $i+j+k$ ungerade ist, was einer negativen Ladung (ungerade Anzahl an Gitterplatz-Schritten bis zum positiv geladenen Referenz-Ion) entspricht und positiver Beitrag, wenn $i+j+k$ gerade ist, was einer positiven Ladung (gerade Anzahl
an Gitterplatz-Schritten bis zum positiv geladenen Referenz-Ion) entspricht). $\alpha$ bezeichnet die sogenannte Madelung-Konstante für einen Natriumchlorid-Kristall. Der Term $i=j=k=0$ ist wegzulassen. Die Bezeichnung dieser Reihe mit einer Konstanten ist gerechtfertigt, da diese
Reihe für $n \to \infty$ konvergiert, wie D. Borwein, J. M. Borwein und K. F. Taylor 1985 zeigten \cite{key3}. Zunächst soll aber angemerkt sein, dass im Allgemeinen die Reihenfolge, in der die Terme summiert werden, eine Rolle bei der Konvergenz spielt. Jedoch zeigten
D. Borwein, J. M. Borwein und K. F. Taylor die Konvergenz der betrachteten Reihe $\alpha$ für expandierende Würfel, was dem physikalischen Sachverhalt eines Natriumchlorid-Kristalls entspricht.\newline Es stellt sich heraus, dass die Darstellung der obigen Reihe nicht allzu geeignet
für die Implementierung ist, da von $-n$ bis $n$ summiert wird. Eine Summation von $1$ bis $n$ ist wesentlich effizienter, weshalb im Folgenden eine entsprechende Formel angegeben wird (Beweis siehe \hyperref[sec:anhang]{Anhang}):
\begin{equation*}
    \sum_{i,j,k=-n}^{n}\frac{(-1)^{i+j+k}}{\sqrt{i^2 + j^2 + k^2}} = 8\sum_{i,j,k=1}^{n}\frac{(-1)^{i+j+k}}{\sqrt{i^2 + j^2 + k^2}} + 12\sum_{i,j=1}^{n}\frac{(-1)^{i+j}}{\sqrt{i^2 + j^2}} + 6\sum_{i=1}^{n} \frac{(-1)^{i}}{i}
\end{equation*}
Für eine endliche Zahl $n$ kann dieser Algorithmus nun mit den entsprechenden Gewichten (Evjen-Methode) implementiert werden. Das Ergebnis ist die Madelung-Konstante für einen Natriumchlorid-Kristall.\\ \\

\noindent 2. Der oben entwickelte Algorithmus wird durch die Benutzung von insgesamt drei for-Schleifen realisiert, die für die Summation über ($i,j,k$), ($i,j$) und ($i$) zuständig sind. Die Gewichtung wird mit entsprechenden if-Statements vorgenommen. In jeder for-Schleife wird das Ergebnis der momentanten Rechnung in einer temporären Variablen gespeichert,
sodass das Endergebnis nach den for-Schleifen durch die obige Formel gegeben ist. Mit dem implementierten Verfahren ergibt sich folgender Wert für die Madelung-Konstante mit einer relativen Genauigeit von $10^{-5}$:
\begin{equation*}
    \alpha = 1.74756
\end{equation*} Ein Vergleich mit dem Literaturwert $\alpha = 1.74756$ \cite{key4} zeigt den Erfolg des implementierten Verfahrens auf. Die Konvergenz für den Wert, welcher sich bei einer relativen Genauigkeit von $10^{-5}$ ergibt, tritt bereits bei $n=47$ (siehe Code) auf.
Mit den Konstanten \cite{key4} $e = 1.602176565 \cdot 10^{-19} \si{\coulomb}$, $\epsilon_0 = 8.854187817 \cdot 10^{-12}\frac{\si{\farad}}{\si{\metre}}$ und $a = 564 \cdot 10^{-12} \si{\metre}$ berechnet sich mit einer relativen Genauigeit von $10^{-5}$ die elektromagnetische Energie eines einzelnen Gitterbausteins im dreidimensionalen Natriumchlorid-Kristall zu:
\begin{equation*}
    V = -4.46176 \si{\electronvolt}
\end{equation*} Hier ist ein Vergleich mit einem Literaturwert nicht sonderlich aussagekräftig, da in den meisten Fällen bei der Berechnung der Energie zusätzlich die kurzreichweitige Abstoßung von zwei Ionen aufgrund des Pauli-Prinzips berücksichtigt wird \cite{key4}. Ein Vergleich mit dem Wert $V = -7.92 \si{\electronvolt}$
aus dem Lehrbuch von R. Gross und A. Marx zeigt dennoch auf, dass das Ergebnis in der richtigen Größenordnung liegt und Abweichungen durch das Ignorieren von weiteren Energiebeiträgen zustande kommen.


\section*{Flacher Kristall}\label{sec:flach}

3. In diesem Aufgabenteil wird die Energie für einen flachen Natriumchlorid-Kristall (immer noch in drei Dimensionen) berechnet. Dementsprechend ist die Summation mit der Zusatzbedingung $j_3 = i_3$ durchzuführen. Dann ist $r_{ij} = a\sqrt{(i_1 - j_1)^2 + (i_2 - j_2)^2}$, womit nach einem analogen Vorgehen wie im ersten Aufgabenteil für die Energie folgt:
\begin{equation*}
    V = \sum_{i,j=-n}^{n} V_{i,j} = \sum_{i,j=-n}^{n} \frac{e}{4\pi\epsilon_0} \frac{e_{i,j}}{a\sqrt{i^2 + j^2}} = \frac{1}{4\pi\epsilon_0} \frac{e^2}{a} \sum_{i,j=-n}^{n} \frac{(-1)^{i+j}}{\sqrt{i^2 + j^2}} = -\frac{1}{4\pi\epsilon_0} \frac{e^2}{a} \alpha
\end{equation*} Auch hier ist natürlich eine Summation von $1$ bis $n$ effizienter:
\begin{equation*}
    \sum_{i,j=-n}^{n} \frac{(-1)^{i+j}}{\sqrt{i^2 + j^2}} = 4\sum_{i,j=1}^{n} \frac{(-1)^{i+j}}{\sqrt{i^2 + j^2}} + 4\sum_{i=1}^{n} \frac{(-1)^{i}}{i}
\end{equation*} Der Beweis dieses Zusammenhangs erfolgt analog wie der angegebene Beweis im \hyperref[sec:anhang]{Anhang}. Auch die Konvergenz dieser Reihe wurde für expandierende Quadrate von D. Borwein, J. M. Borwein und K. F. Taylor bewiesen. Nach der Evjen-Methode werden die Ladungen an den Ecken mit $\frac{1}{4}$ und die Ladungen an Kanten mit $\frac{1}{2}$ gewichtet.
Offensichtlich erfolgt diese Implementierung des Verfahrens analog wie im vorherigen Aufgabenteil, jedoch wird eine for-Schleife weggelassen. Für die Madelung-Konstante ergibt sich dann für eine relative Genauigkeit von $10^{-5}$:
\begin{equation*}
    \alpha = 1.61554
\end{equation*} Damit folgt für die elektromagnetische Energie eines einzelnen Gitterbausteins für einen flachen Natriumchlorid-Kristall:
\begin{equation*}
    V = -4.12469 \si{\electronvolt}
\end{equation*} Hierzu lässt sich kein Vergleich mit einem Literaturwert aufführen, jedoch sind die Ergebnisse besonders auf Grundlage der guten Ergebnisse des vorherigen Aufgabenteiles sehr plausibel.


\section*{Anhang}\label{sec:anhang}

Beweis für die Endformel der Madelung-Konstante aus dem \hyperref[sec:energie]{ersten Abschnitt}:
\begin{align*}
    \sum_{i,j,k=-n}^{n} \frac{(-1)   ^{i+j+k}}{\sqrt{i^2 + j^2 + k^2}} &= \sum_{i=-n}^{n} \sum_{j=-n}^{n} \sum_{k=-n}^{n} \frac{(-1)   ^{i+j+k}}{\sqrt{i^2 + j^2 + k^2}} \\
    &= \sum_{i=-n}^{n} \sum_{j=-n}^{n} \left(\sum_{k=-n}^{-1} \frac{(-1)^{i+j+k}}{i^2 + j^2 + k^2} + \frac{(-1)^{i+j}}{\sqrt{i^2 + j^2}} + \sum_{k=1}^{n} \frac{(-1)^{i+j+k}}{\sqrt{i^2 + j^2 + k^2}}\right) \\
    &= \sum_{i=-n}^{n} \sum_{j=-n}^{n} \left(2\sum_{k=1}^{n} \frac{(-1)^{i+j+k}}{i^2 + j^2 + k^2} + \frac{(-1)^{i+j}}{\sqrt{i^2 + j^2}}\right) \\
    &= \sum_{i=-n}^{n} \left(\sum_{j=-n}^{-1} \left(2\sum_{k=1}^{n} \frac{(-1)^{i+j+k}}{\sqrt{i^2 + j^2 + k^2}} + \frac{(-1)^{i+j}}{\sqrt{i^2 + j^2}}\right) + 2\sum_{k=1}^{n}\frac{(-1)^{i+k}}{\sqrt{i^2 + k^2}} + \frac{(-1)^{i}}{i}\right. \\
    &\left. \quad + \sum_{j=1}^{n}\left(2\sum_{k=1}^{n}\frac{(-1)^{i+j+k}}{\sqrt{i^2 + j^2 + k^2}} + \frac{(-1)^{i+j}}{\sqrt{i^2 + j^2}}\right)\right) \\
    &= \sum_{i=-n}^{n}\left(2\sum_{j=1}^{n}\left(2\sum_{k=1}^{n}\frac{(-1)^{i+j+k}}{\sqrt{i^2 + j^2 + k^2}} + \frac{(-1)^{i+j}}{\sqrt{i^2 + j^2}}\right) + 2\sum_{k=1}^{n}\frac{(-1)^{i+k}}{\sqrt{i^2 + k^2}} + \frac{(-1)^{i}}{i}\right) \\
    &= \sum_{i=-n}^{-1}\left(2\sum_{j=1}^{n}\left(2\sum_{k=1}^{n}\frac{(-1)^{i+j+k}}{\sqrt{i^2 + j^2 + k^2}} + \frac{(-1)^{i+j}}{\sqrt{i^2 + j^2}}\right) + 2\sum_{k=1}^{n}\frac{(-1)^{i+k}}{\sqrt{i^2 + k^2}} + \frac{(-1)^{i}}{i}\right) \\
    &\quad + 2\sum_{j=1}^{n}\left(2\sum_{k=1}^{n}\frac{(-1)^{j+k}}{\sqrt{j^2 + k^2}} + \frac{(-1)^{j}}{j}\right) + 2\sum_{k=1}^{n}\frac{(-1)^k}{k} \\
    &\quad + \sum_{i=1}^{n}\left(2\sum_{j=1}^{n}\left(2\sum_{k=1}^{n}\frac{(-1)^{i+j+k}}{\sqrt{i^2 + j^2 + k^2}} + \frac{(-1)^{i+j}}{\sqrt{i^2 + j^2}}\right) + 2\sum_{k=1}^{n}\frac{(-1)^{i+k}}{\sqrt{i^2 + k^2}} + \frac{(-1)^{i}}{i}\right) \\
    &= 2\sum_{i=1}^{n}\left(2\sum_{j=1}^{n}\left(2\sum_{k=1}^{n}\frac{(-1)^{i+j+k}}{\sqrt{i^2 + j^2 + k^2}} + \frac{(-1)^{i+j}}{\sqrt{i^2 + j^2}}\right) + 2\sum_{k=1}^{n}\frac{(-1)^{i+k}}{\sqrt{i^2 + k^2}} + \frac{(-1)^{i}}{i}\right) \\
    &\quad + 2\sum_{j=1}^{n}\left(2\sum_{k=1}^{n}\frac{(-1)^{j+k}}{\sqrt{j^2 + k^2}} + \frac{(-1)^{j}}{j}\right) + 2\sum_{k=1}^{n}\frac{(-1)^{k}}{k}
\end{align*} Umbenennung von Laufvariablen und Umsortierung der Terme liefert schließlich:
\begin{equation*}
    \sum_{i,j,k=-n}^{n}\frac{(-1)^{i+j+k}}{\sqrt{i^2 + j^2 + k^2}} = 8\sum_{i,j,k=1}^{n}\frac{(-1)^{i+j+k}}{\sqrt{i^2 + j^2 + k^2}} + 12\sum_{i,j=1}^{n}\frac{(-1)^{i+j}}{\sqrt{i^2 + j^2}} + 6\sum_{i=1}^{n} \frac{(-1)^{i}}{i}
\end{equation*} Es wurde dabei benutzt, dass der Term $i=j=k=0$ wegzulassen ist.\newpage

\listoffigures

\bibliographystyle{unsrt}
\bibliography{refs}

\end{document}